\documentclass{article}
\usepackage{amsthm}
\usepackage{xepersian}
\settextfont{XB Zar}


\newtheorem{question}{سوال}
\begin{document}
\section{سرویس‌ها}
یک سرویس در لینوکس برنامه‌ای است که در پشت صحنه\footnote{Background} اجرا می‌شود. اجرای سرویس می‌تواند به صورت خودکار هنگام بالا آمدن سیستم یا به صورت دستی  باشد. 
\subsection{ایجاد سرویس}
\subsection{دستوراتی که می‌تواند درون فایل سرویس قرار گیرد}
\subsubsection{\lr{[UNIT]}}


\subsection{سوالات}
\begin{question}
چرا به سرویس‌ها نیاز داریم؟
\end{question}
\begin{proof}
\end{proof}
\begin{question}

برای ایجاد یک سرویس جدید فایل سرویس را باید در کجا ایجاد کنیم؟
\end{question}
\begin{proof}
\end{proof}
\begin{question}

با چه زبان برنامه نویسی می‌توانیم یک سرویس ایجاد کنیم؟
\end{question}
\begin{proof}
\end{proof}
\begin{question}

تفاوت میان سرویس و \lr{daemon} چیست؟
\end{question}
\begin{proof}
\end{proof}

\end{document}
